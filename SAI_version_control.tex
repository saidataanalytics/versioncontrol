% Options for packages loaded elsewhere
\PassOptionsToPackage{unicode}{hyperref}
\PassOptionsToPackage{hyphens}{url}
%
\documentclass[
]{article}
\usepackage{lmodern}
\usepackage{amssymb,amsmath}
\usepackage{ifxetex,ifluatex}
\ifnum 0\ifxetex 1\fi\ifluatex 1\fi=0 % if pdftex
  \usepackage[T1]{fontenc}
  \usepackage[utf8]{inputenc}
  \usepackage{textcomp} % provide euro and other symbols
\else % if luatex or xetex
  \usepackage{unicode-math}
  \defaultfontfeatures{Scale=MatchLowercase}
  \defaultfontfeatures[\rmfamily]{Ligatures=TeX,Scale=1}
\fi
% Use upquote if available, for straight quotes in verbatim environments
\IfFileExists{upquote.sty}{\usepackage{upquote}}{}
\IfFileExists{microtype.sty}{% use microtype if available
  \usepackage[]{microtype}
  \UseMicrotypeSet[protrusion]{basicmath} % disable protrusion for tt fonts
}{}
\makeatletter
\@ifundefined{KOMAClassName}{% if non-KOMA class
  \IfFileExists{parskip.sty}{%
    \usepackage{parskip}
  }{% else
    \setlength{\parindent}{0pt}
    \setlength{\parskip}{6pt plus 2pt minus 1pt}}
}{% if KOMA class
  \KOMAoptions{parskip=half}}
\makeatother
\usepackage{xcolor}
\IfFileExists{xurl.sty}{\usepackage{xurl}}{} % add URL line breaks if available
\IfFileExists{bookmark.sty}{\usepackage{bookmark}}{\usepackage{hyperref}}
\hypersetup{
  pdftitle={SAI CPD event - Version Control},
  hidelinks,
  pdfcreator={LaTeX via pandoc}}
\urlstyle{same} % disable monospaced font for URLs
\usepackage[margin=1in]{geometry}
\usepackage{graphicx,grffile}
\makeatletter
\def\maxwidth{\ifdim\Gin@nat@width>\linewidth\linewidth\else\Gin@nat@width\fi}
\def\maxheight{\ifdim\Gin@nat@height>\textheight\textheight\else\Gin@nat@height\fi}
\makeatother
% Scale images if necessary, so that they will not overflow the page
% margins by default, and it is still possible to overwrite the defaults
% using explicit options in \includegraphics[width, height, ...]{}
\setkeys{Gin}{width=\maxwidth,height=\maxheight,keepaspectratio}
% Set default figure placement to htbp
\makeatletter
\def\fps@figure{htbp}
\makeatother
\setlength{\emergencystretch}{3em} % prevent overfull lines
\providecommand{\tightlist}{%
  \setlength{\itemsep}{0pt}\setlength{\parskip}{0pt}}
\setcounter{secnumdepth}{-\maxdimen} % remove section numbering

\title{SAI CPD event - Version Control}
\author{}
\date{\vspace{-2.5em}last updated: 2020-07-18}

\begin{document}
\maketitle

{
\setcounter{tocdepth}{2}
\tableofcontents
}
\begin{center}\rule{0.5\linewidth}{0.5pt}\end{center}

\hypertarget{introduction-to-version-control}{%
\section{Introduction to Version Control
}\label{introduction-to-version-control}}

\begin{center}\rule{0.5\linewidth}{0.5pt}\end{center}

\hypertarget{introduction}{%
\subsection{Introduction}\label{introduction}}

Purpose of CPD event:

\begin{itemize}
\tightlist
\item
  Describe what version control systems are and how they can be used by
  Actuaries
\item
  Provide examples of version control systems
\item
  Outline commercial considerations
\item
  Provide a demonstration of how a version control system can be used in
  practice (using Git and Github for demonstration purposes)
\end{itemize}

\hypertarget{what-is-a-version-control-system}{%
\subsection{What is a Version Control
system}\label{what-is-a-version-control-system}}

\begin{itemize}
\tightlist
\item
  Manages the evolution of a set of files (a repository) in a structured
  way
\item
  e.g.~if you want to get technical for a bit, git is a directed acyclic
  graph
  (\url{https://medium.com/girl-writes-code/git-is-a-directed-acyclic-graph-and-what-the-heck-does-that-mean-b6c8dec65059}).
  Might be worth mentioning as something to be aware of since I suspect
  a lot of my ``I had git'' times are when I didn't fully appreciate
  what that means.
\end{itemize}

\hypertarget{benefits-of-using-a-version-control-system}{%
\subsection{Benefits of using a Version Control
system}\label{benefits-of-using-a-version-control-system}}

\begin{itemize}
\tightlist
\item
  Allows professionals to share code with others and make changes which
  can be tracked
\item
  Manages source code history
\item
  Maintains a history of what changes have happened
\item
  Supports collaboration and effective communication between
  professionals (e.g.~Actuaries, Data Scientists, Software Developers)
\item
  Facilitates multiple people working at the same time.
\item
  Helps code hygiene and review processes. Different users can have
  different permissions -e.g.~only senior team members can merge code
  into main branch
\end{itemize}

\hypertarget{examples-of-version-control-systems}{%
\subsection{Examples of Version Control
systems}\label{examples-of-version-control-systems}}

\begin{itemize}
\item
  Git (Global Information Tracker): commercial, distributed version
  control system

  \begin{itemize}
  \tightlist
  \item
    (comment) not sure what you mean by commercial? I didn't actually
    know what git stood for so thank you.
  \item
    (comment) presume you will clarify what you mean by distributed
    version control system
  \end{itemize}
\item
  SVN (SubVersioN): open source, centralised version control system
\item
  CVS (Concurrent Versions System)

  \begin{itemize}
  \tightlist
  \item
    (comment) I hadn't heard of that one either.
  \end{itemize}
\end{itemize}

\hypertarget{local-and-remote-hosting}{%
\subsection{Local and remote hosting}\label{local-and-remote-hosting}}

(comment): I think this needs to go in early on, not necessarily here,
but somewhere. Can discuss further if needed

\begin{itemize}
\tightlist
\item
  will discuss in context of git only (svn is different as it is
  centralised from the start)
\item
  good diagram would be important here, would explain a lot more than
  words.
\item
  central copy of repo (enterprise version or in cloud)

  \begin{itemize}
  \tightlist
  \item
    -\textgreater{} copy down to your local machine (pull)
  \item
    -\textgreater{} make changes/edit/save changes(commit) using git for
    this (iterative process), often several commits
  \item
    -\textgreater{} when ready upload changes to central copy (push)
  \end{itemize}
\item
  discuss branch workflow practices here

  \begin{itemize}
  \tightlist
  \item
    ie all work for a specific self-contained task in a branch so
    isolated from other changes
  \item
    when ready for code to be moved into main branch then push and do a
    \emph{pull request}
  \item
    code reviewed -\textgreater{} merged into master
  \item
    can implement software control practices here like testing, code
    review continuous integration - obviously depends on your job.
  \item
    even for a single user, branch + pull requests + merge can be useful
    - easy to roll back a bad change, contain tasks etc
  \end{itemize}
\end{itemize}

\hypertarget{examples-of-repository-hosting-services}{%
\subsection{Examples of Repository hosting
services}\label{examples-of-repository-hosting-services}}

Github:

\begin{itemize}
\tightlist
\item
  Web based hosting service for Git repositories

  \begin{itemize}
  \tightlist
  \item
    (comment) it is possible to use SVN here too
    (e.g.~\url{https://docs.github.com/en/github/importing-your-projects-to-github/working-with-subversion-on-github})
    though I don't know anyone doing that or how easy it is to do.
    probably easier all round just to use git.
  \end{itemize}
\item
  Free option available with limited functionality

  \begin{itemize}
  \tightlist
  \item
    note that free plans were upgraded a few months ago to incorporate a
    lot more. A key change is that private repositories are now
    available to individuals and organisations.
  \end{itemize}
\item
  Internet Explorer is not supported
\end{itemize}

Other Repository hosting services include:

\begin{itemize}
\tightlist
\item
  Bitbucket (comment): my feeling is that this might be the next most
  popular offering. It offered free private repos long before github did
  IIRC
\item
  Gitlab
\item
  Assembla
\item
  Beanstalk and many more
\end{itemize}

\hypertarget{user-interface-examples}{%
\subsection{User Interface examples}\label{user-interface-examples}}

Git:

\begin{itemize}
\tightlist
\item
  command line git (hardcore users)
\item
  RStudio
\item
  Tortoise (works with git and svn)
\item
  if using github, Github Desktop is easy to use UI
\end{itemize}

\hypertarget{commercial-considerations}{%
\subsection{Commercial considerations}\label{commercial-considerations}}

\begin{itemize}
\tightlist
\item
  TBC
\item
  Security if using cloud based service like github

  \begin{itemize}
  \tightlist
  \item
    data security/privacy - need to ensure things that should not be
    uploaded do not get uploaded
  \end{itemize}
\end{itemize}

\begin{center}\rule{0.5\linewidth}{0.5pt}\end{center}

\hypertarget{demonstration}{%
\section{Demonstration }\label{demonstration}}

\begin{center}\rule{0.5\linewidth}{0.5pt}\end{center}

\hypertarget{introduction-1}{%
\subsection{Introduction}\label{introduction-1}}

For demonstration purposes Git and Github will be used.

Items to be covered in demonstration include:

\begin{itemize}
\tightlist
\item
  Getting set-up
\item
  Create a Repository
\item
  Cloning a Git Repository
\item
  Create a script in R, then Commit and Push to Repository Master branch
\item
  Using gitingore file
\item
  Pull from Repository Master branch, make edits and Push to new branch
\item
  Review and accept or reject edits
\item
  Rename and Delete files
\item
  Recover files
\end{itemize}

\hypertarget{getting-set-up}{%
\subsection{Getting set-up}\label{getting-set-up}}

\begin{itemize}
\tightlist
\item
  Obtaining a Github account
\item
  Downloading Git
\end{itemize}

\hypertarget{creating-a-repository}{%
\subsection{Creating a Repository}\label{creating-a-repository}}

Steps include:

\hypertarget{cloning-a-repository}{%
\subsection{Cloning a Repository}\label{cloning-a-repository}}

Steps include:

\hypertarget{commit-and-push-a-script-from-r}{%
\subsection{Commit and Push a script from
R}\label{commit-and-push-a-script-from-r}}

Steps include:

\hypertarget{pull-and-edit}{%
\subsection{Pull and edit}\label{pull-and-edit}}

Steps include:

\hypertarget{review-and-accept-or-reject}{%
\subsection{Review and Accept or
Reject}\label{review-and-accept-or-reject}}

Steps include:

\hypertarget{rename-and-delete}{%
\subsection{Rename and Delete}\label{rename-and-delete}}

Steps include:

\hypertarget{recover}{%
\subsection{Recover}\label{recover}}

Steps include:

\begin{center}\rule{0.5\linewidth}{0.5pt}\end{center}

\hypertarget{summary}{%
\section{Summary }\label{summary}}

\begin{center}\rule{0.5\linewidth}{0.5pt}\end{center}

\hypertarget{re-cap}{%
\subsection{Re-cap}\label{re-cap}}

\begin{itemize}
\tightlist
\item
  Describe what version control systems are and how they can be used by
  Actuaries
\item
  Provide examples of version control systems
\item
  Outline commercial considerations
\item
  Provide a demonstration of how a version control system can be used in
  practice (using Git and Github for demonstration purposes)
\end{itemize}

\hypertarget{some-useful-links}{%
\subsection{Some useful links}\label{some-useful-links}}

Different Version Control Systems

\begin{itemize}
\tightlist
\item
  \url{https://www.atlassian.com/git/tutorials/what-is-git}
\item
  \url{https://www.perforce.com/blog/vcs/what-svn}
\item
  \url{https://svn.apache.org/repos/asf/openoffice/ooo-site/trunk/content/docs/ddCVS.html.en}
\end{itemize}

Comparison of different Repository Hosting Services

\begin{itemize}
\tightlist
\item
  \url{https://www.git-tower.com/blog/git-hosting-services-compared/}
\end{itemize}

Creating a branch on Github

\begin{itemize}
\tightlist
\item
  \url{https://docs.github.com/en/desktop/contributing-to-projects/creating-a-branch-for-your-work}
\end{itemize}

Merging a branch on Github

\begin{itemize}
\tightlist
\item
  \url{https://docs.github.com/en/github/collaborating-with-issues-and-pull-requests/merging-a-pull-request}
\end{itemize}

\end{document}
